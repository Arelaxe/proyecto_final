\documentclass[a4paper,11pt]{article}
\usepackage[spanish]{babel}
\usepackage[utf8]{inputenc}
\usepackage{hyperref}
\usepackage{listings}
\begin{document}
\title{Estudio de la eficiencia del algoritmo de la burbuja en C++}
\author{Noelia Escalera Mejías}
\date{}
\maketitle
\section{Resumen}
URL del repositorio:
\url{https://github.com/Arelaxe/proyecto_final}
\\ \\
Para el proyecto final del curso de LaTeX y Git, he decidido hacer un estudio sobre la eficiencia del algoritmo de la burbuja, de forma tanto teórica como empírica.
\section{Introducción}
El algoritmo de la burbuja es uno de los primeros algoritmos de ordenación que se aprenden a programar debido a la sencillez de su implementación. Sin embargo, ¿es eficiente? Esto es lo que vamos a comprobar en el presente informe. El lenguaje de programación en el que se trabajará será C++11.
\section{Eficiencia teórica}
Hay varias formas de implementar el algoritmo de la burbuja. Nosotros usaremos la más sencilla:
\lstset{language=C, breaklines=true, basicstyle=\footnotesize}
\begin{lstlisting}[frame=single]
void ordenar(int *v, int n) {
 for (int i=0; i<n-1; i++)
  for (int j=0; j<n-i-1; j++)
   if (v[j]>v[j+1]) {
    int aux = v[j];
    v[j] = v[j+1];
    v[j+1] = aux;
   }
}
\end{lstlisting}
La eficiencia teórica sería la siguiente:

\end{document}